\subsection{Scikit Learn Random Forest (PyRandomForest)}\index{PyRandomForest}
\label{sec:PyRandomForest}

This TMVA method was implemented using RandomForest Classifier 

\href{http://scikit-learn.org/stable/modules/ensemble.html#forests-of-randomized-trees}{Scikit Learn Random Forest}


\subsubsection{Booking options}

PyRandomForest is booked via the command:
\begin{codeexample}
\begin{tmvacode}
factory->BookMethod(dataloader,TMVA::Types::kPyRandomForest, "PyRandomForest","<options>" );
\end{tmvacode}
\caption[.]{\codeexampleCaptionSize Booking of the PyRandomForest classifier: the first argument is 
		   a predefined enumerator, the second argument is a user-defined 
		   string identifier, and the third argument is the configuration options string.
         Individual options are separated by a ':'. 
         See Sec.~\ref{sec:usingtmva:booking} for more information on the booking.}
\end{codeexample}

The configuration options for the PyRandomForest classifier are listed in Option Table~\ref{opt:mva::prf_1}
(see also Sec.~\ref{sec:fitting}).

% ======= input option table ==========================================
\begin{option}[p]
\input optiontables/pymva/MVA__PyRandomForest_1.tex
\caption[.]{\optionCaptionSize 
     Configuration options reference for MVA method: {\em PyRandomForest}.
     Values given are defaults. If predefined categories exist, the default category 
     is marked by a '$\star$'. The options in Option Table~\ref{opt:mva::methodbase} on 
     page~\pageref{opt:mva::methodbase} can also be configured.The table
      is continued in Option Table~\ref{opt:mva::prf_2}.  
}
\label{opt:mva::prf_1}
\end{option}

\begin{option}[p]
\input optiontables/pymva/MVA__PyRandomForest_2.tex
\caption[.]{\optionCaptionSize 
     Continuation of Option Table~\ref{opt:mva::prf_1}.     
}
\label{opt:mva::prf_2}
\end{option}

\begin{option}[p]
\input optiontables/pymva/MVA__PyRandomForest_3.tex
\caption[.]{\optionCaptionSize 
     Continuation of Option Table~\ref{opt:mva::prf_2}.     
}
\label{opt:mva::prf_3}
\end{option}


% =====================================================================

A typical option string could look as follows:
\begin{codeexample}
\begin{tmvacode}
factory->BookMethod(dataloader,TMVA::Types::kPyRandomForest, "PyRandomForest",\newline "!V:NEstimators=200:Criterion=gini:MaxFeatures=auto:MaxDepth=6:MinSamplesLeaf=3:MinWeightFractionLeaf=0:Bootstrap=kTRUE" );
\end{tmvacode}
\label{ce:PyRandomForestexample}
\end{codeexample}

\subsubsection{Description and implementation}
This is an implementation based in Scikit Learn,
read more at \href{http://scikit-learn.org/stable/modules/ensemble.html#forests-of-randomized-trees}{Scikit Learn Random Forest}

\subsubsection{Variable ranking}

The present implementation of PyRandomForest does not provide a ranking 
of the input variables.

% \subsubsection{Performance}
% 
% ....

%%% Local Variables: 
%%% mode: latex
%%% TeX-master: "TMVAUsersGuide"
%%% End: 
